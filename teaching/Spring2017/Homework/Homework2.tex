\documentclass[10pt,a4paper]{article}
\usepackage[utf8]{inputenc}
\usepackage{amsmath}
\usepackage{amsfonts}
\usepackage{amssymb}
\usepackage{graphicx}
\title{PDEs 10422884 -- Homework 2}
\date{}
\begin{document}
\maketitle
\textsl{ This homework must be handed in prior to the tutorial on \textbf{April 27th, 2017}. Questions marked with $*$ will be graded, and will go towards your grade on the homework. Unmarked questions will be checked for completion (or a reasonable attempt).}


\begin{enumerate}

\item[*1.] Given is the PDE $y u_x - x u_y = 0, \, (y>0),$ along with either auxiliary condition (a) $u(x,0) = x^2$ or (b) $u(x,0) = x.$
\begin{enumerate}
\item Reformulate the PDE as a directional derivative, i.e. find the vector $\vec{v}$ such that $\frac{du}{d \vec{v}} = 0$ is equivalent to the PDE. Recall that the PDE then says that solutions $u$ are constant in the direction of this vector $\vec{v},$ i.e. along lines with a slope equal to $\vec{v}.$ Use this geometric idea to find a solution to the PDE in terms of arbitrary function(s), and then attempt to solve the problems (a) and (b).
\item Use the method of characteristics to solve the same boundary value problems. Find the characteristic ODE
\begin{align*}
x_t = a(x,y,u)\\
y_t = b(x,y,u)\\
u_t = c(x,y,u)
\end{align*}
and write each auxiliary condition parametrically in terms of $(s,t).$ Attempt to solve the problems (a) and (b).
\end{enumerate}  

\item[*2.] Use any method (including characteristics and the geometric ideas learned in the tutorial) to solve the constant coefficient equation $2u_t + 3u_x = 0$ subject to the auxiliary condition $u = \sin(x)$ when $t = 0.$

\item[3.] Solve the PDE $a u_x + b u_y + cu = 0$ by using the following new set of coordinates $\xi = ax + by, \, \zeta = bx - ay.$ \emph{Hint:} Use the chain rule to express derivatives w.r.t.\ $x$ as derivatives w.r.t.\ $\xi,$ etc.
\item[4.] Solve the PDE $u_x + u_y = 1$ by using the following new set of coordinates $\xi = x + y, \, \zeta = x - y.$

\item[5.] Solve 
\begin{align*}
& u_t + x u_x = u^3 \\
& u(x,0) = \sin(x)
\end{align*}
As some time $T>0$ the solution $u$ blows up. That is, there exist points $x_0$ such that $|u(x_0,T)| \rightarrow + \infty.$ Find the smallest time $T,$ and the points $x_0$ such that $|u(x_0,T)| \rightarrow + \infty$ as $t \rightarrow T^-.$
\end{enumerate}
\end{document}