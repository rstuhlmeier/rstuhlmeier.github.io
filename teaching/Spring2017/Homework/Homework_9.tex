

\documentclass[10pt,a4paper]{article}
\usepackage[utf8]{inputenc}
\usepackage{amsmath}
\usepackage{amsfonts}
\usepackage{amssymb}
\title{PDEs 10422884 – Homework 9}
\date{}
\begin{document}

\maketitle


This homework must be handed in prior to the tutorial on June 29th,
2017. 
\begin{enumerate}
\item \begin{enumerate} % This is basically the text out of Strauss
\item Consider the eigenvalue problem $X'' = - \lambda X$ on the interval $[0,l]$ with Robin boundary conditions: 
\begin{align*}
X' + X = 0 \text{ on } x = 0,\\
X' = 0 \text{ on } x = l.
\end{align*}
Investigate the problem assuming $\lambda = - \gamma^2 < 0.$ Is/are there one (or many) negative eigenvalue(s)? Explain your answer graphically!
\item Assuming you are treating a wave or heat equation on $[0,l]$, such that the expansion of the solution is 
\begin{equation}
u(x,t) = \sum_n T_n(t) X_n(x) 
\end{equation}
where $X_n(x)$ are the eigenfunctions determined by the boundary conditions, and 
\[ T_n(t) = \begin{cases} A_n e^{-\lambda_n k t} \text{ for the heat eqn.} \\
A_n \cos(\sqrt{\lambda_n}ct) + B_n \cos(\sqrt{\lambda_n}ct) \text{ for the wave eqn.} \end{cases} \]

write down the form of the expansion assuming the Robin boundary conditions given above, and arbitrary initial conditions (e.g.\ $u = \phi$ on $t = 0$ for the heat equation). Try to interpret the results.
\end{enumerate}
\item Solve the forced wave equation % Kazdan solves this in Homework 7
\begin{align*}
& u_{tt} = c^2 u_{xx} + g(x) \sin(\omega t), \, 0 < x < l \\
& u = 0 \text{ on } x = 0, x = l \\
& u = u_t = 0 \text{ on } t = 0.
\end{align*}
For which values of $\omega$ can resonance occur? (Consult your notes from ODEs last semester).
\item Solve via series expansion the heat equation $u_t = u_{xx}$ in $(0,1)$ with $u_x(0,t) = 0, \, u(1,t) = 1,$ and $u(x,0) = x^2.$ Compute the first two coefficients explicitly. What is the equilibrium state (term that does not tend to zero with large time?) % Kazdan solves this in HW 7
\item Write the Legendre equation  % you can find this on wikipedia, or anywhere
\[ (1-x^2) y'' - 2 xy' + n(n+1)y = 0 \]
in Sturm-Liouville form 
\[ (pv')' + q v   = 0 .\]
Do the same for the Bessel equation
\[ r^2 w''(r) + r w'(r) + (r^2 - \nu^2) w(r) = 0.\]
\emph{Hint: try dividing by $r$ and transforming $x = r/\lambda.$}

\item (Optional mathematical background) Recall that we call a relation $\langle \, , \, \rangle:X \times X \rightarrow \mathbb{R}$ from some vector space $X$ into the real numbers an inner product (or scalar product) if $\forall x, \, y, \, x_1, \, x_2 \in X, \, \forall \lambda \in \mathbb{R}:$\\ (a) $\langle x_1 + x_2, y \rangle = \langle x_1,y \rangle + \langle x_2,y \rangle,$ (b) $ \langle \lambda x, y \rangle = \lambda \langle x,y \rangle,$ (c) $\langle x,y \rangle
 = \langle y, x \rangle,$ (d) $\langle x,x \rangle \geq 0,$ (e) $\langle x,x \rangle = 0  \Leftrightarrow x = 0.$ Show, for the vector space of integrable functions on an interval $[a,b] \in \mathbb{R}$ that 
\[ \langle f,g \rangle = \int_a^b f(x) g(x) dx \]
satisfies these properties and defines an inner product on this space.
 \end{enumerate}


\end{document}

