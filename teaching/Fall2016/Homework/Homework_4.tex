\documentclass[a4paper,12pt,leqno]{article}
\usepackage{amsmath,amssymb,amsfonts}
\renewcommand{\baselinestretch}{1.4}


\headheight=0in
\headsep=0in
\footskip=15pt
\textheight=680pt
\topskip=0pt
\topmargin=-10pt



\vfuzz5pt % Don't report over-full v-boxes if over-edge is small
\hfuzz5pt % Don't report over-full h-boxes if over-edge is small
%%%%%%%%%%%%%%%%%%%%%%%%%%%%%%%%%%%%%%%%%%%%%%%%%%%%%%%%%%%%%%%%%%%%%%%%%%%%%%%%%%%%%%%%%%%

\begin{document}
\centering{\Large\textbf{  Ordinary Differential Equations - 10413181}}
\centering{Homework No. 4}

\begin{enumerate}
\item In class we learned about a special type of equation called a \textbf{Bernoulli equation}, which has the general form
\[ y' + a(x) y = b(x) y^n, \quad n \in \mathbb{R}.\]
You know how to solve this equation for $n = 0$ and $n=1$ by general methods. It was shown that a transform $y^{1-n} = v$ can be used to reduce such an equation to a more manageable form. Employ this for the following problems:
\begin{enumerate}
\item $y' - y = y^2$
\item $y' = \frac{2}{x}y + \frac{x}{y^2}$
\item $t^2 y' + 2ty - y^3 = 0$
\end{enumerate}
\item Another special type of equation is known as a \textbf{Riccati equation,} with general form 
\[ y' = q_1(t) + q_2(t)y + q_3(t)y^2. \] \emph{(Note, if $q_1 = 0$ this is just a Bernoulli equation; see if you can spot the commonality in the methods for solving them. Hint: you can always find a certain particular solution to the Bernoulli equation.)} If a particular solution $y_1(t)$ of this equation is known (for example, through educated guesswork), a more general solution can be found via the substitution 
\[ y(t) = y_1(t) + \frac{1}{v(t)}. \]
Try this on the following equations:
\begin{enumerate}
\item $y' = 1+t^2 - 2ty + y^2,$ with particular solution $y_1(t) = t$
\item $y' = -\frac{1}{t^2} - \frac{y}{t} +y^2$ with particular solution $y_1(t) = 1/t$
\end{enumerate}
\item We discussed the matter of existence and uniqueness of solutions in class. Given the initial value problem
\[ y' = y^{1/3} \sin(2t), \quad y(0) = 0 \]
\begin{enumerate}
\item Find a trivial solution.
\item Setting aside the fact that $y(0) = 0$ (you may assume $y(0) = \varepsilon$ and consider the limit, or simply temporarily ignore this) find two other solutions.
\item Check the assumptions of the existence \& uniqueness theorem to explain this multiplicity of solutions.
\end{enumerate}
\end{enumerate}


\end{document}



\end{document}
