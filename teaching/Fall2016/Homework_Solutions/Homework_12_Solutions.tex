%%%%%%%%%%%%%%%%%%%%%%%%%%%%%%%%%%%%%%%%%%%%%%%%%%%%%%%%%
%
%        Ordinary Differential Equations 104131
%           ISE-104131-targil-1.tex
%
%	25.10.2010
%%%%%%%%%%%%%%%%%%%%%%%%%%%%%%%%%%%%%%%%%%%%%%%%%%%%%%%%%
% -------------------------------------------------------
% AMS-LaTeX Paper ***************************************
% -------------------------------------------------------
\documentclass[a4paper,12pt,leqno]{article}
\usepackage{amsmath,amssymb,amsfonts}
\renewcommand{\baselinestretch}{1.4}


\headheight=0in
\headsep=0in
\footskip=15pt
\textheight=680pt
\topskip=0pt
\topmargin=-10pt



\vfuzz5pt % Don't report over-full v-boxes if over-edge is small
\hfuzz5pt % Don't report over-full h-boxes if over-edge is small
%%%%%%%%%%%%%%%%%%%%%%%%%%%%%%%%%%%%%%%%%%%%%%%%%%%%%%%%%%%%%%%%%%%%%%%%%%%%%%%%%%%%%%%%%%%

\begin{document}
\centering{\Large\textbf{  Ordinary Differential Equations - 10413181}}
\centering{  Homework No. 12 -- Solutions}

\begin{enumerate}
\item \[ W(x^1,x^2) = \det \begin{pmatrix}
t & t^2 \\
1 & 2t
\end{pmatrix} = t^2. \]
These are linearly independent everywhere except $t = 0.$
\item \[ W(x^1,x^2) = \det \begin{pmatrix}
t & e^t \\
1 & e^t
\end{pmatrix} = e^t(1-t). \]
These are linearly independent everywhere except $t = 1.$
\item There are two steps to checking that this is a fundamental set of solutions. The first is to check that each of these are in fact \textbf{solutions!} Denoting the matrix by $A$:
\begin{align*}
\frac{d}{dt}x^1 = e^{2t} \begin{pmatrix}
2\\
2\\
2
\end{pmatrix}
= A x^1 \\
\frac{d}{dt}x^2 = e^{-t} \begin{pmatrix}
-1\\
0\\
1
\end{pmatrix}
= A x^2 \\
\frac{d}{dt}x^3 = e^{-t} \begin{pmatrix}
0\\
-1\\
1
\end{pmatrix}
= A x^3 
 \end{align*}
 Subsequently we need to check that they are a fundamental set, for which we employ the Wronskian:
 \[ W(x^1,x^2,x^3) = det\begin{pmatrix}
 e^{2t} & e^{-t} & 0 \\
  e^{2t} & 0 & e^{-t} \\
  e^{2t} & -e^{-t} & -e^{-t} 
    \end{pmatrix}
    = 3 \neq 0 \]

\item For the matrix 
\[A =  \begin{pmatrix}
5 & -1 \\
3 & 1 
\end{pmatrix}\]
we have associated a characteristic polynomial $\lambda^2 - 6\lambda + 8 = 0.$ This has roots $\lambda_1 = 4, \, \lambda_2 = 2.$
Eigenvectors $\mathbf{x}$ to eigenvalue $\lambda_1$ must fulfill $(\mathbf{A} - 2 \mathbf{1})\mathbf{x} = 0$ or 
\[ \begin{pmatrix}
3 & -1 \\
3 & -1 
\end{pmatrix}
\begin{pmatrix}
x_1 \\
x_2
\end{pmatrix}
= 
\begin{pmatrix}
0 \\
0
\end{pmatrix} \]
Hence $3x_1 = x_2$ and an eigenvector is given by
\[ \xi_1 = \begin{pmatrix}
1 \\
3
\end{pmatrix}.
\]
Eigenvectors $\mathbf{x}$ to eigenvalue $\lambda_2$ must fulfill $(\mathbf{A} - 4 \mathbf{1})\mathbf{x} = 0$ or 
\[ \begin{pmatrix}
1 & -1 \\
3 & -3 
\end{pmatrix}
\begin{pmatrix}
x_1 \\
x_2
\end{pmatrix}
= 
\begin{pmatrix}
0 \\
0
\end{pmatrix} \]
Hence $x_1 = x_2$ and another eigenvector is given by
\[ \xi_2 = \begin{pmatrix}
1 \\
1
\end{pmatrix}.
\]
The general solution is then written in the form 
\[ \mathbf{x}(t) = c_1 e^{2t} \begin{pmatrix}
1 \\
3
\end{pmatrix}
+ c_2 e^{4t} \begin{pmatrix}
1 \\
1
\end{pmatrix} \]
for arbitrary constants $c_1, c_2$ to be determined by the initial conditions. 
Given the initial condition 
\[ \mathbf{x}(t=0) = \begin{pmatrix}
4 \\
-2
\end{pmatrix}
\]
we find 
\[  c_1 \begin{pmatrix}
1 \\
3
\end{pmatrix}
+ c_2  \begin{pmatrix}
1 \\
1
\end{pmatrix} = 
\begin{pmatrix}
4 \\
-2
\end{pmatrix}\]
or equivalently
\[ \begin{pmatrix}
1 & 1 \\
3 & 1 
\end{pmatrix}
\begin{pmatrix}
c_1 \\
c_2
\end{pmatrix}
= 
\begin{pmatrix}
4 \\
-2
\end{pmatrix}\]
Inverting the matrix then yields $c_1 = -3$ and $c_2 = 7.$ The solution to the IVP is thus
\[ -3 e^{2t} \begin{pmatrix}
1 \\
3
\end{pmatrix}
+ 7 e^{4t} \begin{pmatrix}
1 \\
1
\end{pmatrix} \]
\end{enumerate}

\end{document}
 
