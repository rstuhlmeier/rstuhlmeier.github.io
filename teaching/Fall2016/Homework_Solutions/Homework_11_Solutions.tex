%%%%%%%%%%%%%%%%%%%%%%%%%%%%%%%%%%%%%%%%%%%%%%%%%%%%%%%%%
%
%        Ordinary Differential Equations 104131
%           ISE-104131-targil-1.tex
%
%	25.10.2010
%%%%%%%%%%%%%%%%%%%%%%%%%%%%%%%%%%%%%%%%%%%%%%%%%%%%%%%%%
% -------------------------------------------------------
% AMS-LaTeX Paper ***************************************
% -------------------------------------------------------
\documentclass[a4paper,12pt,leqno]{article}
\usepackage{amsmath,amssymb,amsfonts}
\renewcommand{\baselinestretch}{1.4}


\headheight=0in
\headsep=0in
\footskip=15pt
\textheight=680pt
\topskip=0pt
\topmargin=-10pt



\vfuzz5pt % Don't report over-full v-boxes if over-edge is small
\hfuzz5pt % Don't report over-full h-boxes if over-edge is small
%%%%%%%%%%%%%%%%%%%%%%%%%%%%%%%%%%%%%%%%%%%%%%%%%%%%%%%%%%%%%%%%%%%%%%%%%%%%%%%%%%%%%%%%%%%

\begin{document}
\centering{\Large\textbf{  Ordinary Differential Equations - 10413181}}
\centering{  Homework No. 11 -- Solutions}

\begin{enumerate}
\item $ y '' + \cos(x) y = 0$ is given.
We make the power series ansatz (about $x_0 = 0$)
\[ y = \sum_{n=0}^\infty a_n x^n, \text{ so that } y'' = \sum_{n=0}^\infty (n+2)(n+1) a_{n+2} x^n. \]
Using the Taylor series of $\cos$ about zero, 
\[ \cos(x) = 1 - \frac{x^2}{2} + \frac{x^4}{4!} - \ldots \]
we see
\[ \sum_{n=0}^\infty (n+2)(n+1) a_{n+2} x^n = - \sum_{n=0}^\infty a_n x^n \left(1 - \frac{x^2}{2} + \frac{x^4}{4!} - \ldots \right). \]
Collecting terms, we find 
\begin{align*}
2 a_2 = - a_0 \\
6 a_3 = - a_1 \\
12 a_4 = a_0 
\end{align*}
which suffices to write down the first four terms of the solution:
\[ y = a_0 + a_1 x - \frac{a_0}{2}x^2 - \frac{a_1}{6}x^3 + \frac{a_0}{12}x^4 + \ldots \]
In principle this could be continued as long as desired.
\item Note \[ \sum_{n=0}^\infty x^n := \lim_{N \rightarrow \infty} \sum_{n=0}^N x^n. \]
Hence 
\begin{align*}
& (1-x) \sum_{n=0}^\infty x^n = (1-x)\lim_{N \rightarrow \infty} \sum_{n=0}^N x^n = \lim_{N \rightarrow \infty} (1-x) \sum_{n=0}^N x^n \\
&= \lim_{N \rightarrow \infty} (1-x) (1 + x + x^2 + \ldots + x^N) \\
&= \lim_{N \rightarrow \infty} (1 + x \ldots + x^N - x - x^2 - \ldots - x^{N+1})\\
&= \lim_{N \rightarrow \infty} (1-x^{N+1}) = 1.
\end{align*}
\item \begin{enumerate}
\item \begin{align*}
& u'' + 2u' + 2u = 0 \\
& v:= u' \Rightarrow v' = u'' = -2u' - 2u = -2v - 2u\\
& \begin{cases}
u' = v \\
v' = -2v -2u
\end{cases}
\end{align*}
\item 
\begin{align*}
& t^2 u'' + tu' + (t^2 - 1) u = 0 \\
& v:= u' \Rightarrow v' = u'' = -\frac{u'}{t} - \frac{(t^2 -1)u}{t^2} = -\frac{v}{t} - \frac{(t^2 -1)u}{t^2} \\
& \begin{cases}
u' = v\\
v' = -\frac{v}{t} - \frac{(t^2 -1)u}{t^2}
\end{cases}
\end{align*}
\item \begin{align*}
& u'''' - u = 0 \\
& u' =:v,\quad v'=:w=u'',\quad w'=:z=u''',\quad z'=u''''=u \\
&\begin{cases}
u' = v\\
v' = w\\
w' = z\\
z' = u
\end{cases}
\end{align*}
\end{enumerate}
\item The system is equivalent to the following in matrix form:
\[ \begin{pmatrix}
1 & -2 & 3 \\
-1 & 1 & -2 \\
2 & -1 & -3 
\end{pmatrix}
\begin{pmatrix}
x_1 \\
x_2 \\
x_3
\end{pmatrix}
= 
\begin{pmatrix}
7 \\
-5 \\
4
\end{pmatrix}. \]
Inverting the matrix leads to 
\[ x_1 = 2\dfrac{1}{3}, \quad x_2 = -1 \dfrac{1}{3}, \quad x_3 = \dfrac{2}{3}. \]
\item \begin{enumerate}
\item The characteristic polynomial associated with matrix A is $\lambda^2 - \lambda - 2 = 0$ which has roots 2 and -1.
\item The characteristic polynomial associated with matrix B is $\lambda^2 - 2 \lambda - 35 = 0$ which has roots 7 and -5.
\end{enumerate}
\end{enumerate}

\end{document}
 
