\documentclass[10pt,a4paper]{article}
\usepackage[utf8]{inputenc}
\usepackage{amsmath}
\usepackage{amsfonts}
\usepackage{amssymb}
\usepackage{graphicx}
\title{Homework 5 Solutions}
\date{}
\begin{document}

\maketitle

\noindent \textbf{1. Parameter--dependent ODEs and bifurcation}

Given the first order, nonlinear, separable ODE
\[ \frac{dx}{dt} = r + x^2 \]
with a parameter $r \in \mathbb{R},$ we have several options of investigating the behavior of solutions. We shall see that the most straightforward (namely solving the equation) is the hardest and gives the least insight. This is a standard situation, and one reason why qualitative analysis of ODEs (especially those for which an explicit solution cannot be found) is so valuable. 

\noindent \emph{Step 1: Solving the ODEs}
\[ x' = r + x^2 \Rightarrow \int \frac{dx}{r+x^2} = t + C \]
\begin{enumerate}
\item ( $r = -a^2 <0$)
\begin{equation} \int \frac{dx}{x^2 - a^2} = 
\begin{cases} 
\frac{1}{2a} \ln(\frac{x-a}{x+a}) & x^2 > a^2 \\
\frac{1}{2a} \ln(\frac{a+x}{a-x}) & x^2 < a^2
\end{cases}
\end{equation}
Hence either  
\[ \frac{x-a}{x+a} = Ce^{2at} \Rightarrow x = a \frac{1+Ce^{2at}}{1-Ce^{2at}} \]
or 
\[ \frac{a+x}{a-x} = Ce^{2at} \Rightarrow x = a \frac{Ce^{2at}-1}{1+Ce^{2at}} \]
\item ($ r = 0$)
\begin{equation}
\int \frac{dx}{x^2} = -x^{-1} 
\end{equation}
Hence 
\[ x = \frac{1}{C - t} \]
\item ($r = a^2 > 0$)
\begin{equation}
\int \frac{dx}{x^2 + a^2} = \frac{1}{a} \arctan(x/a) 
\end{equation}
Hence
\[ x = a \tan(at + C) \]
\end{enumerate}
It is rather hard to get information about the dependence of each of these solutions on the parameter $r.$ Instead, we may take a different approach: first we shall plot the right-hand side of the equation, $f(y) = r+x^2$ for the three cases:
\begin{figure}
\centering
\includegraphics[scale=0.5]{TwoFP}
\caption{$r<0$}
\end{figure}
\begin{figure}
\centering
\includegraphics[scale=0.5]{SingleFP}
\caption{$r = 0$}
\end{figure}
\begin{figure}
\centering
\includegraphics[scale=0.5]{NoFP}
\caption{$r > 0$}
\end{figure}

\newpage

Note that each zero of $f(y)$ is a \emph{equilibrium point} where $x' = 0!$ We can easily see that decresing the parameter from $r = 1$ (where there are no equilibrium solutions) creates a single equilibrium point when $r = 0,$ which further splits off into two equilibrium points for $r < 0.$ These basic dynamics are captured from the direction fields:  


\begin{figure}[h]
\centering
\includegraphics[scale=0.4]{DirFieldTwoFP}
\caption{$r<0$}
\end{figure}

\begin{figure}[h]
\centering
\includegraphics[scale=0.4]{DirFieldOneFP}
\caption{$r = 0$}
\end{figure}
\begin{figure}[h!]
\centering
\includegraphics[scale=0.4]{DirFieldNoFP}
\caption{$r > 0$}
\end{figure}
 \newpage

This phenomenon of fixed points appearing or disappearing (or stability shifting) is known as bifurcation. Note that the fixed point appearing at $r = 0$ is semi-stable, and of the two fixed points appearing after bifurcation one is stable and the other unstable.

For more information, look at the excellent book \emph{Nonlinear Dynamics and Chaos} by Steven Strogatz.

\begin{enumerate}
\item[Problem 2.]
\begin{enumerate}
\item \begin{align*}
r^2 + 3r - 4 = 0 \Rightarrow r_1 = 1, r_2 = -4 \\
\Rightarrow y = C_1 e^t + C_2 e^{-4t} 
\end{align*}
\item \begin{align*}
y'' + 5y' = 0 \Rightarrow r^2 + 5r = 0 \Rightarrow r_1 = 0, r_2 = -5\\
\Rightarrow y = C_1 e^{-5t} + C_2
\end{align*}
\item \begin{align*}
r^2 + 3r + 2 = 0 \Rightarrow r_1 = -1, r_2 =  -2 \\
y = C_1 e^{-t} + C_2 e^{-2t}
\end{align*}
\item \begin{align*}
r^2 - 3r + 2 = 0 \Rightarrow r_1=1, r_2 =2\\
\Rightarrow y = C_1 e^t + C_2 e^{2t} \\
y(0) = C_1 + C_2 = 1 \\
y'(0) = C_1 + 2C_2 = 1 \\
\Leftrightarrow C_1 = 1, C_2 = 0 \Rightarrow y = e^t
\end{align*}
\item \begin{align*}
r^2 + 3r = 0 \Rightarrow r_1 = -3, r_2 = 0 \\
\Rightarrow y = C_1 e^{-3t} + C_2 \\
y(0) = C_1 + C_2 = -2 \\
y'(0) = -3 C_1 = 3 \Rightarrow C_1 = -1 \Rightarrow C_2 = -1\\
\Rightarrow y = -e^{-3t} - 1
\end{align*}
\end{enumerate}
\end{enumerate}
\end{document}